\documentclass[11pt,a4paper]{article}
\bibliographystyle{apalike}
\usepackage{amssymb}
\usepackage{epsfig}
\usepackage{amsmath}
\usepackage{natbib}
\usepackage[margin=0.4in]{geometry}
\begin{document}

\section{Optimal Quadratic Estimators}
Given a model covariance matrix $C=S+N$ including both the signal and
noise covariance, we can form the optimal quadratic estimator for some
paramters $\mathbf{\theta}$ by:
\begin{align}
  F_{mn} &=\frac{1}{2}{\rm Tr}\left[\mathbf{C}^{-1}\mathbf{C}_{,m}\mathbf{C}^{-1}\mathbf{C}_{,n}\right]\nonumber \\
  q_n &=\frac{1}{2}{\rm Tr}\left[\mathbf{C}^{-1}\mathbf{C}_{,n}\mathbf{C}^{-1}\mathbf{N}\right]\nonumber \\
  f_n &=\frac{1}{2}\mathbf{y}^{\rm
    T}\mathbf{C}^{-1}\mathbf{C}_{,n}\mathbf{C}^{-1}\mathbf{y}.\nonumber \\
  \hat{\theta}_m &
  =\sum\limits_N\left(F^{-1}\right)_{mn}\left(q_n-f_n\right) \label{eqn:estimator}
\end{align}
Here, $\mathbf{y}$ is the full data covariance matrix (think of this
as making a single long vector from the two measured timeseries
$C(t_i)$ and $L(t_i)$). $\mathbf{C}$ is the full covariance of this
vector, and the $\mathbf{C}_{,n}$ notation is meant to indicate the
derivative of the covariance matrix with respect to the $n^{\rm th}$
parameter we are trying to measure.

\section{The Problem}
We observe two light curves for quasars -- $C(t)$, the continuum, and
$L(t)$, the line emission.

We believe that $C(t)$ is that it is a stationary random process best
modeled as a damped random walk parameterized by $\sigma$ and $\mu$,
such that:
\begin{equation}
  {\rm cov}\left(C(t_i),C(t_j)\right)=\xi_{CC}=\sigma^2e^{-\frac{|t_i-t_j|}{\mu}}
\end{equation}
We allow that the light curves are observed at irregular times, $t_i$,
but insist that $C$ and $L$ are always observed simultaneously.

The idea for $L(t)$ is that line emission lags the continuum by some
time that depends (possibly straightforwardly) on the BH mass, and
since the broad-line-emitting region (BLR) is larger than the
continuum-emitting region (CR), the BLR emission is also less
synchronized. We represent this lag and desynchronicity by convolution
with a transfer function $\psi(t)$ such that:
\begin{equation}
  L(t) = \int_{-\infty}^{\infty}C(t')\psi(t'-t)\, dt'
\end{equation}
If, for example, the BLR was an exact copy of the CR light curve, but
with some lag $\tau$, then we would have $\psi(t)= \delta(t-\tau)$.
We will parameterize $\psi$, until we have a good reason not to, as a
set of binned amplitudes, with bin width $w$:
\begin{equation}
  \psi(t)=\sum\limits_m \psi_m \Pi\left(t-t_m|w\right)
  \label{eqn:psiparam}
\end{equation}
where $\Pi(t|w)$ is a top hat function of unit area and width $w$.

\section{Covariance Matrices}
In order to make use of equation~\ref{eqn:estimator}, we need to calculate the
full data covariance matrix. For our purposes, the observations
are independent and the errors are known, so it is the signal
covariance matrix between observation times $S_{ij}=S(t_i-t_j)$  that
needs to be calculated. Fortunately, this can be done analytically for
both timeseries. 

${\rm cov}(C(t_i),C(t_j))$ is known by hypothesis, and fortunately the
DRW correlation function is analytically tractable, so the BLR-CR and
BLR-BLR correlations can, with some effort, be written down.


\subsection{BLR covariance}
To write the BLR covariance:
\begin{align}
C_{LL}(t) & = \int_{-\infty}^{\infty}L(t_i)L(t_i-(t_i-t_j)) dt_i \\
& = \int_{-\infty}^{\infty}dt_i\,\,\left[\int_{-\infty}^{\infty}dt'\,\,C(t')\psi(t'-t_i)\right]\left[\int_{-\infty}^{\infty}dt''\,\,C(t'')\psi(t''-t_i+t)\right]
\end{align}
For the above and what follows, we will write the unprimed $t$ for
$t_i-t_j$.  Taking the Fourier Transform (FT) of this expression with
respect to $t$ causes $\psi \mapsto \hat{\psi}e^{-i\omega
  \left(t''-t_i\right)}$. The additional phase factor turns the other
integrals into either FTs or inverse FTs, and we are left with:
\begin{equation}
  = |\hat{\psi}|^2|\hat{C}|^2
\end{equation}
Inverse-transforming this expression, we have that the line emission
covariance is simply the convolution of the CR correlation function
with a new quantity, $\xi_{\psi}$:
\begin{align}
  \xi_{\psi} = \int_{-\infty}^{\infty}dt'\,\psi(t')\psi(t'-t) \\
  C_{LL}(t)= \xi_{LL} = \int_{\infty}^{\infty}dt' \xi_{CC}(t')\xi_{\psi}(t'-t)
\end{align}
Next, we hope to write $\xi_{LL}$ in terms of our chosen parameterization, given
by equation~\ref{eqn:psiparam}. Doing this requires evaluating the
following integrals: 
\begin{equation}
\xi_{\psi}^{mn}(t)=
\psi_m\psi_n\int_{-\infty}^{\infty}dt'\,\Pi(t'-t_m|w)\Pi(t'-t-t_n|w)
  \label{eqn:xipsi}
\end{equation}
\begin{equation}
C_{LL}^{mn}=\int_{-\infty}^{\infty}dt'\,\xi_{CC}(t')\xi_{\psi}^{mn}(t'-t)
\label{eqn:cll_integral}
\end{equation}
The integral in equation~\ref{eqn:xipsi} is straightforward -- the
convolution of identical unit-area top hat functions of width $w$ (and
height $w^{_-1}$) is an isosceles triangle centered at the origin of
height $w^{-1}$ and base $2w$, which I'll denote as $T(x|w)$:
\begin{equation}
T(x|w)= \begin{cases}
  \frac{1}{w}-\frac{|x|}{w^2} & \text{ for $|x| \le w $} \\
  0 & \text{else}
\end{cases}
\end{equation}
Now the autocorrelation of the binned transfer function with itself
is:
\begin{equation}
  \xi_{\psi}^{mn}(t)= \psi_m\psi_n\,T(t-(t_m-t_n)|w)
\end{equation}
We can now perform the integral in equation~\ref{eqn:cll_integral}.
\begin{equation}
C_{LL} = \sum\limits_{m,n}\psi_m\psi_n\int_{-\infty}^{\infty}dt'\,\,\sigma^2e^{-\frac{|t'|}{\mu}}T((t'-t)-(t_m-t_n)|w)
\end{equation}
First, make the integration variable swap $t'\mapsto z =
(t'-t)-(t_m-t_n)$, and note that the new limits of integration are $-w
\le z \le w$. This can be done analytically; there are four cases.
\begin{equation}
  C_{LL}=
  \sum\limits_{m,n}\psi_m\psi_ne^{-\frac{|t+(t_m-t_n)}{\mu}}\times  
  \begin{cases}
    \frac{\mu^2}{w^2}e^{-\frac{w+y}{\mu}}\left(e^{\frac{w}{\mu}}-1\right)^2
    & \text{for $y > w$} \\
    \frac{\mu^2}{w^2}e^{-\frac{y-w}{\mu}}\left(e^{\frac{w}{\mu}}-1\right)^2
    & \text{for $y < -w$} \\
    \frac{\mu}{w^2}e^{-\frac{w+y}{\mu}}\left(-\mu -
      \mu e^{2\frac{y}{\mu}}+2\mu e^{\frac{w+2y}{\mu}}-2w
      e^{\frac{w+y}{\mu}} - 2ye^{\frac{w+y}{\mu}}\right) & \text{for
      $-w < y \leq 0$} \\
    \frac{\mu}{w^2}e^{-\frac{y+w}{\mu}}\left( \mu - 2\mu
      e^{\frac{w}{\mu}} +\mu e^{2\frac{y}{\mu}} + 2w
      e^{\frac{y+w}{\mu}} - 2y e^{\frac{y+w}{\mu}} \right) & \text{for
      $0 < y \leq w$}
  \end{cases}
\end{equation}
This is one part of the covariance matrix which appears (along with
its $\psi$-derivatives) in the estimator in
equation~\label{eqn:estimator}.


\subsection{CR-BLR covariance}
To write the cross-covariance:
\begin{align}
{\rm cov}(C(t_i),L(t_j))& = \int_{-\infty}^{\infty}dt'C(t_i)\int_{-\infty}^{\infty}C(t'')\psi(t''-t_i-t)
\end{align}
As before, taking the Fourier transform with respect to $t$ produces
$\psi\mapsto \hat{\psi}e^{-i\omega(t''-t_j)}$. This converts the
remaining integrals into Fourier and inverse Fourier transforms; the
covariance matrix in frequency space is now:
\begin{equation}
  \hat{C}_{CL}(\omega) = |\hat{C}|^2\hat{\psi}
\end{equation}
Performing the inverse transform gives us the expression for the
time-domain covariance:
\begin{equation}
C_{CL}(t) = \int_{-\infty}^{\infty}\,dt'\,\, \xi_{CC}(t')\psi(t'-t)
\end{equation}
Re-writing $\psi$ in its parameterized form, we have:
\begin{equation}
C_{CL}(t) = \sum\limits_{m}\, \psi_m\,\int_{-\infty}^{\infty}\,dt'\,\,\xi_{CC}(t')\Pi(t'-t-t_m|w)
\end{equation}
Substituting in the expression for $\xi_{CC}$ yields:
\begin{equation}
C_{CL}(t) = \sum\limits_{m}\, \sigma^2\psi_m\,\int_{-\infty}^{\infty}\,dt'e^{-\frac{|t'|}{\mu}}\Pi(t'-t-t_m|w)
\end{equation}
The convolution with the top-hat filter can be done analytically,
yielding an expression for $C_{CL}$:
\begin{equation}
  C_{CL}=\sum\limits_{m}\psi_m\sigma^2\times
  \begin{cases}
    \frac{\mu}{w}e^{-\frac{w+y}{\mu}}\left(e^{2\frac{w}{\mu}}-1\right)
    & \text{for $y > w$} \\
    -\frac{\mu}{w}e^{-\frac{w+y}{\mu}}\left(1+e^{2\frac{y}{w}}-2e^{\frac{y+w}{\mu}}\right)
    & \text{for $|y|\leq w$} \\
    2\frac{\mu}{w}e^{\frac{y}{\mu}}\sinh\left(\frac{w}{\mu}\right) &
    \text{for $y < w$}
  \end{cases}
\end{equation}
Note that, when I actually implementated, I found that this expression
was just about exactly a factor of two too large; for
$\psi_m=\delta_{m,0}$, the zero-lag covariance should be equal to
$C_{CC}$, which is $\sigma^2$. 


\subsection{Covariance Matrix Summary}
The full signal covariance matrix $S$ can be written:
\begin{equation}
S = \left(\begin{matrix}
C_{CC} & C_{CL} \\
C_{CL} & C_{LL}
\end{matrix}\right)
\label{eqn:Smatrix}
\end{equation}
where the expressions for the covariances matrices are, for the sake
of neatness and completeness:
\begin{align}
  C_{CC,ij}& = \sigma^2e^{-\frac{|t_i-t_j|}{\mu}} \\
  C_{CL}&=\sum\limits_{m}\psi_m\sigma^2\times
  \begin{cases}
    \frac{\mu}{w}e^{-\frac{w+t_i - t_j - t_m}{\mu}}\left(e^{2\frac{w}{\mu}}-1\right)
    & \text{for $t_i - t_j - t_m > w$} \\
    -\frac{\mu}{w}e^{-\frac{w+t_i - t_j - t_m}{\mu}}\left(1+e^{2\frac{t_i - t_j - t_m}{w}}-2e^{\frac{t_i - t_j - t_m+w}{\mu}}\right)
    & \text{for $|t_i - t_j - t_m|\leq w$} \\
    2\frac{\mu}{w}e^{\frac{t_i - t_j - t_m}{\mu}}\sinh\left(\frac{w}{\mu}\right) &
    \text{for $t_i - t_j - t_m < w$}
  \end{cases}\\
   C_{LL}& =
  \sum\limits_{m,n}\psi_m\psi_ne^{-\frac{|t+(t_m-t_n)}{\mu}}\times  \\
&  \begin{cases}
    \frac{\mu^2}{w^2}e^{-\frac{w+(t_i-t_j) + (t_m-t_n)}{\mu}}\left(e^{\frac{w}{\mu}}-1\right)^2
    & \text{for $(t_i-t_j) + (t_m-t_n) > w$} \\
    \frac{\mu^2}{w^2}e^{-\frac{(t_i-t_j) + (t_m-t_n)-w}{\mu}}\left(e^{\frac{w}{\mu}}-1\right)^2
    & \text{for $(t_i-t_j) + (t_m-t_n) < -w$} \\
    \frac{\mu}{w^2}e^{-\frac{w+(t_i-t_j) + (t_m-t_n)}{\mu}}\left(-\mu -
      \mu e^{2\frac{(t_i-t_j) + (t_m-t_n)}{\mu}}+2\mu e^{\frac{w+2(t_i-t_j) + (t_m-t_n)}{\mu}}-2w
      e^{\frac{w+(t_i-t_j) + (t_m-t_n)}{\mu}} - 2\left[(t_i-t_j) + (t_m-t_n)\right]e^{\frac{w+(t_i-t_j) + (t_m-t_n)}{\mu}}\right) & \text{for
      $-w < (t_i-t_j) + (t_m-t_n) \leq 0$} \\
    \frac{\mu}{w^2}e^{-\frac{(t_i-t_j) + (t_m-t_n)+w}{\mu}}\left( \mu - 2\mu
      e^{\frac{w}{\mu}} +\mu e^{2\frac{(t_i-t_j) + (t_m-t_n)}{\mu}} + 2w
      e^{\frac{(t_i-t_j) + (t_m-t_n)+w}{\mu}} - 2\left[(t_i-t_j) + (t_m-t_n)\right] e^{\frac{(t_i-t_j) + (t_m-t_n)+w}{\mu}} \right) & \text{for
      $0 < (t_i-t_j) + (t_m-t_n) \leq w$}
  \end{cases}
\end{align}



\section{Constructing the Estimator.}
In principle, now, everything is in place. The derivatives of $S$ with
respect to the parameters $\psi_m$ are easy to compute -- they
collapse the sum over $m$ to a single term in each of the cases
above. 

What's notable about this is that, as written, {\it you need to know
  the answer to compute the estimator} -- the form of $\hat{\psi}$
depends explicitly on $\psi$. But that's actually okay -- what people
do with power spectra estimators is guess the answer, $\psi_{\rm
  guess}$, and then use that as the first of two iterations; the
$\psi_{\rm guess}$ is updated after one pass, and then the best guess
is used to build the real estimator.

I don't think it's useful to try to write the analytic form for
$C^{-1}$ -- rather, this part should just be done numerically. The
next step is to try this on simulated data.



\end{document}
